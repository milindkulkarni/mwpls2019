\newif\ifdraft\draftfalse

\newcommand{\poplyear}{2017}
\newcommand{\poplloc}{Paris, France}
\newcommand{\plmwnum}{Sixth}
\newcommand{\numNSFstudents}{8} %Normally 15, but 10 this year because we are carrying over leftovers
\newcommand{\popldates}{January~18 to January~20}
\newcommand{\plmwdate}{January~17}
\newcommand{\plmwurl}{\url{http://conf.researchr.org/home/PLMW-2017}}
\newcommand{\hotelperstudent}{\$750}
\newcommand{\registration}{\$400}
\newcommand{\foodperstudent}{\$100}
\newcommand{\airfareperstudent}{\$1,250}
\newcommand{\costperstudent}{\$2,500}
\newcommand{\industrysupport}{between \$20K and \$30K}
\newcommand{\industrysponsors}{Facebook, Google, IBM Research, Jane Street Capital, and SIGPLAN}
\newcommand{\plmworganizers}{Loris D'Antoni, Eva Darulova, Alexandra Silva, and Dimitris Vytiniotis}
\newcommand{\plmwspeakers}{%not including the panels of young researchers
Amal Ahmed, Rajeev Alur, Andrew Appel, Tom Ball, Sandrine Blazy, Adam Chlipala, Charles Consel,
Byron Cook, Isil Dillig, Derek Dreyer,
Jean-Christophe Filliatre, Nate Foster, Philippa Gardner, Sumit Gulwani, Robert Harper,
John Hughes, Simon Peyton Jones, Shriram Krishnamurthi, Xavier Leroy,
Paul McKenney, Matt Might, Greg Morrisett, Peter M\"uller,
Peter O'Hearn, Benjamin Pierce, Frank Pfenning, Damien Pous, John Reppy, Davide Sangiorgi, Peter Sewell,
Viktor Vafeiadis, Jan Vitek, Philip Wadler, Stephanie Weirich, and Steve Zdancewic}
\newcommand{\citeplmws}{\cite{plmw12,plmw13,plmw14,plmw15,plmw16}}

\documentclass[11pt]{article}
\usepackage[margin=1in,includefoot]{geometry}

\usepackage{color}
\usepackage{tikz}
\usepackage{mathtools}
\usepackage{mathpartir}
\usepackage{amsmath}
\usepackage{amssymb}
\usepackage{url}
\usepackage{bnf}

\newcommand{\parahead}[1]{\paragraph{#1}}
\def\mwpls{Midwest Programming Languages Summit}

\begin{document}

\section*{Summary} 

{\Large 2019 Midwest Programming Languages Summit (MWPLS)}

\parahead{Overview}

The Midwest features many research-intensive institutions that have large and growing presences in the fields of programming languages

We plan to organize a research workshop, called the Midwest PL Summit,
to bring together researchers and students from the greater Midwest
region. In contrast to more formal conference and workshop venues
organized by ACM and IEEE, this informal research workshop will serve
as a forum for researchers and students to share in-progress
research ideas and receive timely feedback to influence subsequent
work.
This will be particularly valuable
because of the high concentration of programming languages researchers
in the Midwest, and because it will provide graduate student
researchers opportunities to network and further develop their
research and presentation skills.

\vspace{8pt}

\parahead{Intellectual Merit}

The main focus of our one-day event will be to present and share
research ideas. Our event will include approximately fifteen
presentations on specific research projects, as well as a poster
session that will accommodate many more. These talks and posters will
span a variety of timely and relevant topics in programming languages
research, and will focus on in-progress and early-stage results, providing an opportunity to receive feedback at a level not available at traditional venues. We expect this forum to lead to the further development of results that will subsequently appear at high-quality workshops, conferences, and journals.

\vspace{8pt}

\parahead{Broader Impacts}

%% Programming languages research, both the foundations and applications
%% in software engineering, is an important source of ideas and
%% techniques for the software technology that pervades our society.
This proposal will foster the programming languages research community in
the Midwest region, by providing networking opportunities for
researchers to develop collaborations across organizations and for
students to develop connections with senior researchers and other
graduate students that will help with their future careers.
By strengthening
the intellectual ties among researchers in the Midwest region, the
graduate students will develop a community that will support them in
their subsequent careers in research, industry, or other sectors. The
workshop will also bring together students and researchers from a
variety of organizations (research-focused universities, industrial
research labs, and teaching-focused colleges), which will help new
research ideas flow into classroom settings and vice versa.

\parahead{Key Words} student travel support, mentoring,
diversity

\newpage
\pagenumbering{arabic}
\setcounter{page}{1}

\section{Introduction}

The Midwest features many research-intensive institutions that have large and growing presences in the fields of programming languages, formal methods, and software engineering. Researchers at these schools are targeting areas such as
type systems and program logics for
software verification; high-performance compiler implementations for
parallel and multi-core hardware; tools and techniques for software
engineering, web application security; and the application of
programming language technology to diverse settings such as quantum
computing and human-computer interaction.

These researchers, which include faculty students, postdocs, and research scientists, often interact with each other at workshops and conferences (such as those organized by the ACM and IEEE) and through formal research collaborations. However, there is substantial benefit to providing a forum for a substantial portion of these researchers to interact with one another in a more informal research gathering. Such a forum allows for the exchange of ideas on in-progress research, provides networking opportunities which can lead to future graduate school positions or jobs, and fosters a sense of community that can spur further collaboration between the critical mass of researchers and across the large number of schools in the Midwest.

\parahead{\mwpls{}: Past editions}

The Midwest PL Summit's first edition was in 2015 at Purdue University\footnote{~\url{http://purdue-pl.github.io/PLSummit/}}. Subsequent editions were held at the University of Chicago (2016), Indiana University (2017), and the University of Wisconsin (2018). In 2019, we are planning to bring the event back to Purdue University. For each of these editions, participation ranged between 80 and 100 attendees, with representation from 8-12 universities. These editions each featured roughly a dozen research presentations as well as a poster session to encourage even more participation of researchers (especially students). Participant details are provided in Table~\ref{tab:participants}.

\begin{table}[hb]
\centering
\begin{tabular}{lccc}
{\bf Location} & {\bf Year} & {\bf \# of Participants} & {\bf \# of Students} \\
\hline
Purdue University & 2015 & 86 & 70\\
University of Chicago & 2016 & $\sim$100 & $\sim$80\\
Indiana University & 2017 & 97 & 72\\
University of Wisconsin & 2018 & 107 & 80
\end{tabular}
\caption{Details of past participation in Midwest PL Summit.}
\label{tab:participants}
\end{table}


%In the last four years, PL researchers at Purdue University, University of Chicago, and University of Indiana organized the first three
%editions of the Midwest PL Summit to serve as an informal programming
%languages research gathering for the greater Midwest
%region.\footnote{~\url{http://purdue-pl.github.io/PLSummit/}} The
%event was a success. In the first edition, approximately 90 people attended (including
%$\approx{}$80 students and $\approx{}$10 faculty) from more than 4
%different universities; 9 research presentations were delivered; and
%$\approx{}$10 faculty research overview presentations were delivered.
%Similar numbers followed in later editions.
%This gathering complemented the successful series of Midwest
%Verification Days over the past several
%years.\footnote{~\url{http://publish.illinois.edu/midverday15/}}.
%Although there is some overlap between the Verification and PL
%communities, the scope of topics in the PL Summit attracted and served
%a distinct population of researchers.

\section{Proposed Activities: Midwest PL Summit 2019}

Based on the success of last years' events, we plan to organize the
fifth edition of the Midwest PL Summit later this year in West Lafayette, hosted again by Purdue University.

\parahead{Dates and Participants}

The workshop will take place on Monday, September 23.
We expect to have participants
from more than 15 research universities, liberal arts colleges, and
industrial research labs, including:
Grinnell College,
DePaul,
DePauw University,
Google,
Indiana University,
Loyola University Chicago,
Mozilla Research,
Northwestern University,
Purdue University,
Ohio State University,
University of Chicago,
University of Illinois at Urbana-Champaign,
University of Illinois at Chicago,
University of Iowa,
University of Kansas,
University of Wisconsin at Madison, and
University of Michigan at Ann Arbor.
We are expecting the number
of participants to be 100 or more.

\parahead{Venue}

We have secured a venue for presentations, coffee breaks, meals, and the poster session, at the Shively Club at Purdue University. This is a venue used by many large workshops hosted at Purdue each year. The venue is located on campus at Purdue University. Purdue is centrally located in the Midwest, and is within easy driving distance of a large number of the universities and colleges we expect to participate. The campus is also located about an hour drive from Indianapolis International Airport. We have secured blocks of rooms at several nearby hotels to accommodate travelers who wish to stay overnight. Our
workshop website will provide information about transportation
options, as well as hotel accommodation for attendees that choose to
stay overnight before or after the event.

\parahead{Meeting Format}

We are planning a one-day event with the goal of maximizing
opportunities for researchers (students, in particular) to
present and share research ideas. Our tentative schedule is the
following:

\begin{itemize}
\item 08:30--09:30: Breakfast
\item 09:30--12:00: Research Presentations
\item 12:00--01:30: Lunch
\item 01:30--03:00: Research Presentations
\item 03:00--04:00: Poster Session and Coffee Break
\item 04:00--05:30: Research Presentations
\item 05:30--: Reception and Dinner (optional)
\end{itemize}

%% \begin{enumerate}
%% \item[(a)] a series of approximately ten to fifteen research presentations, and
%% \item[(b)] a poster session.
%% \end{enumerate}

\noindent
The schedule will allow for approximately fifteen research
presentations, as well as a poster session.
To select the program of talks and posters, we will distribute an open
call for talk and poster submissions through our workshop web site. We
will aim to choose talks that cover a variety of topics that are
likely to find broad interest among the audience, and we will favor
to provide speaking opportunities for students, particularly those who have
not presented in past events. We expect to accommodate all of the poster
submissions.

In addition to the research program, our schedule includes
opportunities for informal and unstructured conversations and
networking: breakfast (before the morning session), lunch (in between
the morning and afternoon sessions), and a coffee break to coincide
with the poster session. After the day's activities, we will organize
an optional social dinner event in the area.

\section{Benefit for Students}

Over the past four years, students have derived several benefits from MWPLS. In particular, the forum gives students a low-stakes, but high quality, setting to present in-progress or early-stage research. The large number of prominent researchers from across the Midwest means that the students will receive substantial, meaningful feedback. Further, the event allows students to build their research networks, both of other students, as well as of faculty and senior researchers.

MWPLS also hosts undergraduate students from across the region, exposing them to the research conducted at the area's schools, and allowing them to make connections in pursuit of future graduate school opportunities.

\section{Outcome of past support from NSF}

The NSF supported last year's MWPLS, held at the University of Wisconsin. The table below summarizes the results of that funding:

\begin{tabular}{cccccc}
	Year & Venue & PI & Applicants & Awarded & \# URM \& women \\
	\hline
	2018 & U of Wisconsin & Loris D'Antoni & 46 & 30 & 10
\end{tabular}

\section{Student Travel Support Details}

\subsection{Need for student travel support}

Although many students will be able to travel to and from the event
within a single day, many 
students will not. For example, the commute from universities 
in Iowa is four or five hours. Students traveling from far away will
likely need to stay overnight in West Lafayette the day before and/or the
day after the event.

\subsection{Spending plan}

We expect to distribute 15-30 such awards to cover the costs of accommodation and defray the costs of travel. We expect this primarily to cover the cost of accommodation (which will be further defrayed by encouraging sharing rooms with other students attending the summit).

\subsection{Recruitment process}

We have begun advertising the event by directly contacting programming
languages researchers from a variety of organizations in the Midwest
(described earlier) to collect general preferences about the workshop. We will create a public web page for the event and distribute the information widely
(e.g., by contacting researchers and students directly, and by posting
announcements on public mailing lists and on Twitter).

\subsection{Outreach to underrepresented groups}

In addition to our general publicity for the event, we will reach out specifically to institutions that serve underrepresented groups (e.g., MSIs). In his role as co-organizer of the Programming Languages Mentoring Workshop, co-PI Kulkarni has curated a list of contacts at these institutions, and will reach out to those in the greater Midwest region.

\subsection{Application process}

Any student enrolled full-time at an accredited
university or college is welcome to apply for a grant.
To apply for an awards, we will ask students
to submit information about how far they will have to travel to attend
the event, what modes of transportation they have access to, and
whether or not their travel expenses would be covered by their home
institutions or not. Applications will be collected at registration time. 

\subsection{Selection criteria}

Preference will be given to: 

\begin{itemize}
\item Students presenting at the workshop
\item Students from under-represented groups
\item Students whose research interests align closely with PL topics
\item Students who would not otherwise be able to attend due to
financial limitations
\end{itemize}

After determining students' needs based on these
factors, we will distribute the travel award funds to
these students. We will also endeavor to distributed the funds broadly across students traveling from different institutions

\section{Anti-Harassment Policies}

Purdue has a formal, publicly available, anti-harassment policy: \\\url{https://www.purdue.edu/policies/ethics/iiic1.html}. This policy details both the university (and members') commitments to combating harassment, as well as reporting information.

%\section{Publicity and Web Page}
%
%We have begun advertising the event by directly contacting programming
%languages researchers from a variety of organizations in the Midwest
%(described earlier) to collect general preferences about the workshop. We will create a public web page for the event and distribute the information widely
%(e.g., by contacting researchers and students directly, and by posting
%announcements on public mailing lists and on Twitter).
%
%\parahead{Call for Participation}
%
%Our web page will provide a public and open call for
%research talk and poster submissions, due approximately one
%month before the event. At that point, the organizers will review all
%of the submissions and select a program of talks as described earlier.
%
%\parahead{Local Arrangements}
%
%Our web page will also provide information about various
%transportation options (car, train, and bus) to and from Madison,
%as well as information about local travel and accommodation near the
%workshop venue on the University of Wisconsin campus.
%
%\parahead{Archiving Presentations}
%
%After the event, we will give talk and poster presenters the
%opportunity to upload their presentations to be linked from the event
%web page for persistent, public access.
%
%\section{Selection of Student Travel Awards}
%
%Although many students will be able to travel to and from the event
%within a single day, many 
%students will not. For example, the commute from universities 
%Iowa is four or five hours. Students traveling from far away will
%likely to need to stay overnight in West Lafayette the day before and/or the
%day after the event.
%
%\parahead{Student Travel Awards}
%
%To help subsidize the costs to attend, our budget includes funding for
%student travel awards. Any student enrolled full-time at an accredited
%university or college is welcome to apply for a grant.
%To apply for an awards, we will ask students
%to submit information about how far they will have to travel to attend
%the event, what modes of transportation they have access to, and
%whether or not their travel expenses would be covered by their home
%institutions or not. Preference will be given to: 
%
%\begin{itemize}
%\item Students presenting at the workshop
%\item Students whose research interests align closely with PL topics
%\item Students from under-represented groups
%\item Students who would not otherwise be able to attend due to
%financial limitations
%\end{itemize}
%
%After determining students' needs based on these
%factors, we will distribute the travel award funds to
%these students. We expect to distribute $\sim$20 such awards to cover the costs of accommodation and defray the costs of travel.
%%For example, if there are 15 students who apply for travel
%%support, we would provide an average stipend of \$100 to each student to help
%%cover travel expenses.
%%The stipend will be paid directly to the students to reimburse their expenses.
%%We will not ask students to provide receipts and detailed breakdowns of
%%the incurred expenses.
%
%\parahead{Transportation and Hotel Sharing}
%
%To further help defray the costs, we will use our event web page to
%promote and facilitate opportunities for students to share travel
%expenses if desired---for example, by carpooling from different
%universities in the same city, or by sharing a hotel room.

\section{Reporting}

After the event, we will submit a summary to NSF with statistics about
the number of attendees (students and senior researchers), the number
of institutions represented, and the research topics and talks
presented. We will also report the number of students who applied for
travel funding, as well as a summary of how the funds were
distributed to students who demonstrated financial need, as described
earlier.

The report will include
the final schedule of the workshop as well as specific details about
the awards given (including name, institution, under-represented group,
citizenship/residency, program of study, interests and stage.)

Shortly after the workshop, we will administer
surveys to collect feedback from all of the workshop participants,
including those funded by NSF and those funded from other sources.
This information will be used to improve future renditions of the workshop.

Later in the year, we will administer a survey to determine what influence the workshop has had on attendees.
Some particular topics of interest are:
\begin{itemize}
\item how did the workshop help fostering collaborations
\item how did PLMW affect research abilities, both technical and communicative
\end{itemize}


\section{Broader Impacts of the Proposed Activities}

Programming languages research, both the foundations and applications
in software engineering, is an important source of ideas and
techniques for the software technology that pervades our society.
The proposed activities will help facilitate the career development of
young computer science researchers pursuing programming languages
research. The workshop will provide opportunities for these students
to develop their presentation skills, which is crucial for translating
ideas from academic communities into practical technologies that can
be used by society at large. Research gatherings are also crucial for
the dissemination and cross-pollination of ideas. By promoting such an
event specifically for young researchers, the collaborations formed at
this event will strengthen their research efforts and community in the
rest of their graduate school careers, as well as in their subsequent
careers in research, industry, and other sectors.
NSF funding in particular will enable students to attend the workshop
who would otherwise not be able to.






\newpage

\setcounter{page}{1}
\bibliographystyle{plain}

\section*{Budget Justification}

Our proposed budget includes two items:

\begin{itemize}

\item \$5,000 will be used to provide travel awards to 15-30 students who
might not otherwise be able to attend the event due to the financial
costs (such as transportation and accommodation).

\end{itemize}

\noindent
With the support of funds from the university and industry partners, we expect not to charge a
registration fee for any of the attendees. We have secured
a workshop venue at the university with the support of the departments of Electrical and Computer Engineering and Computer Science. This venue will host the presentations, poster session, and the summit dinner. Attendees
will be expected to cover their travel costs (transportation to and
from West Lafayette and any hotel fees). The proposed budget will be used to defray travel costs for students who might otherwise not be able to attend.


\newpage

\setcounter{page}{1}

\section*{Data Management Plan}

The data produced as a part of this project will include registration
information and data about attendees, presentation materials, results
of participants surveys, and our final report.

\paragraph*{Data and Metadata Standards}
%
All project data will be stored in standard formats such as
spreadsheets, presentations, PDF, TeX, ASCII, and CSVs. We plan to
bundle this data into a single archive and make it available to future
organizers with documentation in the form of a README file.

\paragraph*{Access and sharing}
%
Aggregate results from participant surveys will be included in a final
report. We plan to make a version of this report (without detailed
information about participants which will be included in our final
report to the NSF) available on the website. The survey results themselves will be
kept confidential. With permission, we plan also to make the
presentations from the workshop available on the web.

\paragraph*{Re-use and re-distribution}
%
We expect that ownership of presentation materials will reside with
the speakers themselves.

\paragraph*{Archiving and Preservation}
%
During the course of the project, we will store data on the workshop
website. After the workshop, we will assemble an archive file make it
available to future organizers.

\newpage

\setcounter{page}{1}
\bibliography{proposal}

\end{document}
